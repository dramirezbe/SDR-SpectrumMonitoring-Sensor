\documentclass[a4paper, 12pt]{article}
\usepackage[utf8]{inputenc}
\usepackage[spanish]{babel}
\usepackage{graphicx}
\usepackage{float}
\usepackage{geometry}
\usepackage{hyperref}
\usepackage{xcolor}

% Configuración de márgenes
\geometry{left=2.5cm, right=2.5cm, top=2.5cm, bottom=2.5cm}

% Estilo de párrafos
\setlength{\parindent}{0pt}
\setlength{\parskip}{0.8em}

\title{\textbf{Manual de Despliegue: Nodo de Monitoreo Espectral}}
\author{Departamento de Ingeniería}
\date{\today}

\begin{document}

\maketitle

\section{Introducción}
Este documento guía paso a paso el proceso para conectar y encender el equipo de monitoreo. El sensor es una unidad diseñada para operar de forma autónoma una vez conectada a la corriente y a la antena.

\subsection{Vista General del Equipo}
A continuación se presenta una vista general del dispositivo ensamblado para su identificación.

\begin{figure}[H]
    \centering
    \includegraphics[width=0.7\textwidth]{img/sensor/vista-general.jpg}
    \caption{Vista general del nodo de monitoreo.}
\end{figure}

\subsection{Componentes Internos y Conectividad}
Componentes principales del equipo:
\begin{itemize}
    \item \textbf{Unidad Principal:} Computador interno (Raspberry Pi 5).
    \item \textbf{Radio:} Receptor interno para capturar señales.
    \item \textbf{Conectividad:} Módulo 4G/LTE (tarjeta SIM) y GPS integrado.
    \item \textbf{Puertos:} Conectores para antenas externas.
\end{itemize}

\begin{figure}[H]
    \centering
    \includegraphics[width=0.7\textwidth]{img/sensor/vista-interna.jpg}
    \caption{Vista interna: Ubicación del módulo de comunicaciones y ranura SIM.}
\end{figure}

% -------------------------------------------------------------------
% FASE 1: HARDWARE (Tono Sencillo)
% -------------------------------------------------------------------
\section{Fase 1: Instalación Física}

\subsection{Paso 1: Conexión de Antena y Energía}
Antes de configurar el software, realice las conexiones físicas en el siguiente orden. Identifique primero los puertos de conexión del dispositivo:

\begin{figure}[H]
    \centering
    \includegraphics[width=0.6\textwidth]{img/sensor/conectores-usb-eth.jpg}
    \caption{Detalle de conectores: USB-C (Energía) y Ethernet.}
\end{figure}

\begin{enumerate}
    \item \textbf{Antena:} Enrosque el cable de la antena en el conector del equipo. Por defecto se usa el \textbf{Puerto 1} (el conector de más arriba). Apriételo con la mano suavemente hasta que no gire más.
    \item \textbf{Encendido:} Conecte el cable USB-C a la entrada de energía.
    
    \begin{quote}
        \textbf{IMPORTANTE:} Si al conectar el cable las luces no encienden después de 10 segundos, desconecte el cable del puerto, \textbf{gírelo (déle la vuelta al conector)} y vuelva a insertarlo.
    \end{quote}
    
    \item \textbf{Confirmación:} Espere unos 20 segundos. El equipo estará listo cuando la luz roja parpadee de forma constante.
\end{enumerate}

\begin{figure}[H]
    \centering
    \includegraphics[width=0.7\textwidth]{img/sensor/antenas.jpg}
    \caption{Ubicación de los puertos de antena.}
\end{figure}

\subsection{Paso 2: Identificación del Equipo}
Para registrar el equipo necesita su identificador único. 

Busque la etiqueta pegada en la caja y anote el código llamado \textbf{MAC} (son letras y números separados por dos puntos, ej. \texttt{d0:65:...}). Necesitará este dato exacto para el siguiente paso.

\begin{figure}[H]
    \centering
    \includegraphics[width=0.6\textwidth]{img/sensor/etiqueta.jpg}
    \caption{Ejemplo de la etiqueta con el código MAC.}
\end{figure}

% -------------------------------------------------------------------
% FASE 2: SOFTWARE
% -------------------------------------------------------------------
\section{Fase 2: Configuración en el Sistema}

\subsection{Paso 3: Ingreso a la Plataforma}
Desde su computador, abra el navegador y entre a:

\begin{center}
    \large\textbf{\url{http://rsm.ane.gov.co:1280/}}
\end{center}

En el menú de la izquierda, vaya a \textbf{Red de monitoreo}, luego "Sensores"\ y haga clic en el botón \textbf{+ Nuevo}.

\begin{figure}[H]
    \centering
    \includegraphics[width=0.7\textwidth]{img/frontend/paso1.png}
    \caption{Botón para agregar sensor.}
\end{figure}

\subsection{Paso 4: Registro}
Llene el formulario con los datos del equipo:

\begin{itemize}
    \item \textbf{MAC:} Escriba el código de la etiqueta (Fase 1).
    \item \textbf{Ubicación:} Ingrese la Latitud y Longitud donde instaló el equipo.
    \item \textbf{Estado:} Seleccione ``Activo''.
    \item Haga clic en \textbf{Crear Sensor}.
\end{itemize}

\begin{figure}[H]
    \centering
    \includegraphics[width=0.6\textwidth]{img/frontend/paso2.png}
    \caption{Formulario de registro.}
\end{figure}

\subsection{Paso 5: Verificar Conexión}
Vaya a la sección \textbf{Análisis} en el menú lateral.
\begin{enumerate}
    \item En "Fuente de datos", elija: \textbf{Tiempo Real (API)}.
    \item En la lista "Sensor", busque el nombre de su equipo.
    \item Verifique que aparezca un punto \textbf{verde}. Esto significa que el equipo tiene internet y está respondiendo.
\end{enumerate}

\begin{figure}[H]
    \centering
    \includegraphics[width=0.7\textwidth]{img/frontend/paso3.png}
    \caption{Verificación de estado en línea.}
\end{figure}

% -------------------------------------------------------------------
% PARTE TÉCNICA (DSP)
% -------------------------------------------------------------------
\subsection{Paso 6: Parámetros de Adquisición DSP}
Configure los parámetros de procesamiento digital de señales (DSP) para definir la ventana de captura espectral.

Definición técnica de variables:
\begin{itemize}
    \item \textbf{Frecuencia Central ($f_c$):} Define el centro exacto de la banda a analizar (ej. 97.5 MHz).
    \item \textbf{Sample Rate ($f_s$):} Define la tasa de muestreo del SDR. Debe ser coherente con la capacidad del bus de datos (Para HackRF, 20 MS/s).
    \item \textbf{Span:} Ancho de banda instantáneo visualizado. En arquitecturas de conversión directa (Zero-IF), este valor suele ser igual al Sample Rate.
    \item \textbf{RBW (Resolution Bandwidth):} Ancho de banda de resolución. Determina el tamaño del ``bin'' en la transformada rápida de Fourier (FFT). Un RBW menor reduce el piso de ruido (mejor SNR) pero incrementa la carga computacional.
\end{itemize}

Haga clic en \textbf{Iniciar Adquisición} para enviar la orden al sensor e iniciar el streaming de datos IQ/FFT.

\begin{figure}[H]
    \centering
    \includegraphics[width=0.7\textwidth]{img/frontend/paso4.png}
    \caption{Panel de control de ingeniería RF.}
\end{figure}

\end{document}