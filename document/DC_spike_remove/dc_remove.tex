\documentclass{article}
\usepackage{amsmath}
\usepackage{amssymb}
\usepackage{booktabs}
\usepackage{geometry}
\geometry{a4paper, margin=1in}

\title{\textbf{Methods for DC Spike Removal in Software Defined Radios (SDR)}}
\author{}
\date{}

\begin{document}

\maketitle

\section{Introduction}
In Direct Conversion (Zero-IF) receivers like the HackRF One, a "DC Spike" appears at $0$ Hz in the baseband spectrum. This artifact results from Local Oscillator (LO) leakage and static DC offsets in the Analog-to-Digital Converters (ADCs).

Mathematically, the received baseband signal $y[n]$ containing the ideal signal $x[n]$ and a complex DC offset $D$ is modeled as:
\begin{equation}
    y[n] = x[n] + D
\end{equation}
where $D = I_{dc} + jQ_{dc}$ is a static or slowly varying constant.

\section{Method 1: Digital Averaging (DC Blocking / Notch Filter)}
This is the most common software implementation (e.g., the "DC Remove" checkbox in GQRX). It estimates the DC component by averaging the signal over time and subtracting it.

\subsection{Block Averaging}
For a block of $N$ samples, the estimated DC offset $\hat{D}$ is:
\begin{equation}
    \hat{D} = \frac{1}{N} \sum_{n=0}^{N-1} y[n]
\end{equation}
The clean signal is then:
\begin{equation}
    y_{clean}[n] = y[n] - \hat{D}
\end{equation}

\subsection{IIR High-Pass Filter (Moving Average)}
In real-time streaming, a First-Order Infinite Impulse Response (IIR) filter is often used to track slowly drifting DC offsets. The recursive update for the estimated DC level $\hat{D}[n]$ is:
\begin{equation}
    \hat{D}[n] = \alpha \cdot y[n] + (1 - \alpha) \cdot \hat{D}[n-1]
\end{equation}
where $0 < \alpha \ll 1$ determines the "forgetting factor" or notch width. The output is:
\begin{equation}
    y_{clean}[n] = y[n] - \hat{D}[n]
\end{equation}
\textbf{Drawback:} This effectively places a spectral notch at DC ($0$ Hz), destroying any signal information located at the center frequency.

\section{Method 2: Offset Tuning (The "Gold Standard")}
Offset tuning avoids the DC spike entirely by shifting the hardware tuning frequency away from the target signal, then digitally shifting the spectrum back. This preserves the signal at the center frequency.

\subsection{Step 1: Hardware Tuning with Offset}
If the target frequency is $f_c$, the hardware LO is tuned to:
\begin{equation}
    f_{LO} = f_c - f_{offset}
\end{equation}
The down-converted analog signal $x_{baseband}(t)$ will now contain the target signal centered at $+f_{offset}$, while the DC spike remains at $0$ Hz (the LO frequency).

\subsection{Step 2: Digital Frequency Translation}
The ADC samples this signal. To recover the target signal at $0$ Hz (baseband), the software applies a digital complex mixer:
\begin{equation}
    y_{centered}[n] = y[n] \cdot e^{-j 2\pi \frac{f_{offset}}{f_s} n}
\end{equation}
where $f_s$ is the sampling rate.

\textbf{Result:} The target signal moves to $0$ Hz. The physical DC spike (originally at $0$ Hz) is shifted to $-f_{offset}$. Since the spike is now outside the signal bandwidth, it can be easily filtered out or ignored.

\section{Method 3: Blind Source Separation (IQ Balancing)}
While primarily used for IQ imbalance (image rejection), algorithms like Recursive Least Squares (RLS) can inherently correct DC offsets if the cost function penalizes non-zero means.

The corrected signal $y_{corr}[n]$ is computed from the raw $I$ and $Q$ inputs using adaptive weights $w$:
\begin{equation}
    y_{corr}[n] = w_1[n] I[n] + w_2[n] Q[n] + C_{dc}[n]
\end{equation}
where $C_{dc}[n]$ is an adaptive bias weight that tracks and cancels the DC offset dynamically without a fixed notch filter.

\section{Comparison of Methods}

\begin{table}[h]
\centering
\begin{tabular}{@{}llll@{}}
\toprule
\textbf{Method} & \textbf{Mechanism} & \textbf{Signal Loss?} & \textbf{Computation Cost} \\ \midrule
\textbf{DC Blocking (Avg)} & Subtracts mean & Yes (Notch at DC) & Very Low \\
\textbf{Offset Tuning} & Frequency Shift & \textbf{No (Lossless)} & Low (Complex Mix) \\
\textbf{Adaptive Filter} & Statistical estimation & No & High \\ \bottomrule
\end{tabular}
\caption{Comparison of DC Removal Techniques}
\end{table}

\end{document}